\begin{longtable}{l|l|l} \caption{Variables used in the machine learning models.} \label{tab:Table1} \\ 
\hline 
\textbf{Variables}  & \textbf{Type}  & \textbf{Source}  \\ 
 \hline 
Date  & Temporal  & \begin{tabular}[c]{@{}l@{}}PM2.5 monitoring \\data\end{tabular}  \\ 
 \hline 
\begin{tabular}[c]{@{}l@{}}Coordinates in degrees (Latitude \\and Longitude)\end{tabular}  & Spatial  & \begin{tabular}[c]{@{}l@{}}PM2.5 monitoring \\data\end{tabular}  \\ 
 \hline 
\begin{tabular}[c]{@{}l@{}}Count of Active Fire Points within \\each of the following radial \\buffers (25 km, 50 km, 100 \\km, and 500 km) for same day \\and up to 7 lag days for each \\radial buffer\end{tabular}  & Spatiotemporal  & \begin{tabular}[c]{@{}l@{}}MODIS Thermal Anomalies/Fire \\Daily L3 Global \\1km product\end{tabular}  \\ 
 \hline 
\begin{tabular}[c]{@{}l@{}}Binary Fire indicator (0 for no \\active fire points in any buffer \\radii or lag for given point; \\1 otherwise)\end{tabular}  & Spatiotemporal  & \begin{tabular}[c]{@{}l@{}}MODIS Thermal Anomalies/Fire \\Daily L3 Global \\1km product\end{tabular}  \\ 
 \hline 
\begin{tabular}[c]{@{}l@{}}Summed length (in meters) of arterial \\(A) and collector (C) \\roads and both (A + C) within \\100, 250, 500, and 1000 m buffer \\radii\end{tabular}  & Spatial  & \begin{tabular}[c]{@{}l@{}}National Highways Planning \\Network \end{tabular}  \\ 
 \hline 
Population Density  & Spatial  & \begin{tabular}[c]{@{}l@{}}**Ellen, can you provide \\this information?\end{tabular}  \\ 
 \hline 
AOD (unitless)  & Spatiotemporal  & \begin{tabular}[c]{@{}l@{}}MODIS Terra and Aqua \\combined Multi-angle \\Implementation of \\Atmospheric Correction \\(MAIAC) dataset\end{tabular}  \\ 
 \hline 
\begin{tabular}[c]{@{}l@{}}planetary boundary layer height \\(surface) (m)\end{tabular}  & Spatiotemporal  & NAM  \\ 
 \hline 
\begin{tabular}[c]{@{}l@{}}temperature at 2m above ground \\(degrees Celsius)\end{tabular}  & Spatiotemporal  & NAM  \\ 
 \hline 
\begin{tabular}[c]{@{}l@{}}relative humidity at 2m above \\ground (%)\end{tabular}  & Spatiotemporal  & NAM  \\ 
 \hline 
\begin{tabular}[c]{@{}l@{}}dew point temperature at 2m above \\ground (degrees Celsius)\end{tabular}  & Spatiotemporal  & NAM  \\ 
 \hline 
APCP.surface  & Spatiotemporal  & NAM  \\ 
 \hline 
WEASD.surface  & Spatiotemporal  & NAM  \\ 
 \hline 
SNOWC.surface  & Spatiotemporal  & NAM  \\ 
 \hline 
UGRD.10.m.above.ground  & Spatiotemporal  & NAM  \\ 
 \hline 
VGRD.10.m.above.ground  & Spatiotemporal  & NAM  \\ 
 \hline 
mean sea level pressure   & Spatiotemporal  & NAM  \\ 
 \hline 
surface pressure  & Spatiotemporal  & NAM  \\ 
 \hline 
DZDT.850.mb  & Spatiotemporal  & NAM  \\ 
 \hline 
DZDT.700.mb  & Spatiotemporal  & NAM  \\ 
 \hline 
TimeZone  & Spatial  & \begin{tabular}[c]{@{}l@{}}**are we using this \\as a \\predictor variable???\end{tabular}  \\ 
 \hline 
\begin{tabular}[c]{@{}l@{}}% of urban development within \\each of the following radial buffers \\(1 km, 5 km, and 10 km)\end{tabular}  & Spatial  & \begin{tabular}[c]{@{}l@{}}National Land Cover \\Database\end{tabular}  \\ 
 \hline 
NDVI  & Spatiotemporal  & MODIS   \\ 
 \hline 
\begin{tabular}[c]{@{}l@{}}indicator variables for season \\(**put in how we \\decide the seasons)\end{tabular}  & Temporal  & derived from date  \\ 
 \hline 
indicator variables for state  & Spatial  & \begin{tabular}[c]{@{}l@{}}derived from latitude \\and longitude\end{tabular}  \\ 
 \hline 
Day of Week  & Temporal  & derived from date  \\ 
 \hline 
Cosine of Day of Week  & Temporal  & derived from date  \\ 
 \hline 
Cosine of Day of Year  & Temporal  & derived from date  \\ 
 \hline 
Indicator variable for year  & Temporal  & derived from date  \\ 
 \hline 
elevation  & Spatial  & NED  \\ 
 \hline 
\end{longtable} 
