\subsection{Visible Infrared Imaging Radiometer Suite (VIIRS) (VNP14IMGTDL\_NRT) }
\subsubsection*{Data Source}
\begin{itemize}[nolistsep]
\item \textbf{Contact}
\item \textbf{Citation/Link}
\item \textbf{Data (local)}
\item \textbf{Geographic Extent}
\item \textbf{Temporal Extent}
\item \textbf{Acknowledgment}
\end{itemize}
\subsubsection*{Brief Description}

We will collect data about fire detection locations, size, and fire radiative power from the Visible Infrared Imaging Radiometer Suite (VIIRS) (VNP14IMGTDL\_NRT) 
\citep{Schroeder2014}. % not sure if that's the right citation
Using GIS techniques, we will create daily clusters of fire points and use these to calculate: (1) the distance to the nearest fire cluster by day and (2) the sum of Fire Radiative Power (FRP) of the nearest clusters of fires by day as it is likely that smoke levels are higher closer to fires. The MODIS product spans longer than our study period (2008-2014) at daily temporal resolution and has a spatial resolution of 1 km. VIIRS was launched in 2011 and has 12 h temporal resolution with 750 m resolution. The BAECV can detect fires larger than 4 km\textsuperscript{2} and provides an estimate of the date of the fire and is available from 1984-2015. 

\subsubsection*{Notes}
\subsubsection*{File Format} .csv
\subsubsection*{Data Filtering and Processing}
\subsubsection*{Final Variable(s)}
\subsubsection*{Methods}
\begin{enumerate}
\item Go to the \href{https://firms.modaps.eosdis.nasa.gov/download/}{NASA EarthData Fire Information for Resource Management System (FIRMS) online tool} and navigate to the Archive section. Click 'Create New Request' and specify spatial and temporal resolution. Also choose 'VIIRS' from Fire Data Source. Choose 'csv' as file type and enter email address. Wait for email, which will contain a .zip file with the data.
\end{enumerate}
\subsubsection*{Quality Control}
\subsubsection*{Script Names}
\begin{enumerate}
\item n/a
\end{enumerate}
\subsubsection*{Data File Names}
\begin{enumerate}
\item fire\_archive\_V1\_2770.csv
\end{enumerate}
