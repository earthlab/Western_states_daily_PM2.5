\subsection{Elevation}
\subsubsection*{Data Source}
\begin{itemize}[nolistsep]
\item \textbf{Contact}
\item \textbf{Citation/Link}
\item \textbf{Data (local)}
\item \textbf{Geographic Extent}
\item \textbf{Temporal Extent}
\item \textbf{Acknowledgment}
\end{itemize}
\subsubsection*{Brief Description}

Elevation can influence PM\textsubscript{2.5} concentrations; for example, PM\textsubscript{2.5} can accumulate in mountain valleys during persistent cold air pools 
(commonly referred to as inversions) 
during winter \citep{Whiteman2014}. We will get elevation data from the 3D Elevation Program, which has resolution of 1 arc-second. This resolution is approximately 10 m north/south and varies east/west with latitude \citep{USGSElevation2017}.

\subsubsection*{Notes}
\subsubsection*{File Format}
\subsubsection*{Data Filtering and Processing}
\subsubsection*{Final Variable(s)}
\subsubsection*{Methods}
\begin{enumerate}
\item Navigate to the \href{https://viewer.nationalmap.gov/basic/?basemap=b1&category=ned,nedsrc&title=3DEP%20View}{National Map Viewer} site and find products for Elevation Products (3DEP), 1 arc-second DEM, IMG file format. Once results are returned, select "Save as Text", which will download a text file containing server links to each NED tile.
\item Download the data using the \href{https://github.com/earthlab/estimate-pm25/blob/master/download-earth-observations/NED/download_tiles.py}{download\_tiles.py} script, which will access the text file that you just downloaded.
\item Extract the elevation values using the \href{https://github.com/earthlab/estimate-pm25/blob/master/download-earth-observations/NED/extract_values_to_points.py}{extract\_values\_to\_points.py} script.
\end{enumerate}
\subsubsection*{Quality Control}
\subsubsection*{Script Names}
\begin{enumerate}
\item download\_tiles.py
\item extract\_values\_to\_points.py
\end{enumerate}
\subsubsection*{Data File Names}
\begin{enumerate}
\tiem n/a
\end{enumerate}