\subsection{\texorpdfstring{National Highways Planning Network}

\subsubsection*{Data Source}

\begin{itemize}[nolistsep]
\item \textbf{Contact} 
\item \textbf{Citation/Link} \url{https://www.fhwa.dot.gov/planning/processes/tools/nhpn/index.cfm} 
\item \textbf{Data (local)} 
\item \textbf{Geographic Extent} Entire USA (including Alaska, Hawaii)
\item \textbf{Temporal Extent} Current
\item \textbf{Acknowledgment} 
\end{itemize}

\subsubsection*{Brief Description}
"Geospatial network database that contains line features representing just over 450,000 miles of highways in the United States"
... "In addition to the NHS [National Highway System], the NHPN covers all roads functionally classified as principal arterial and rural minor arterial."

\subsubsection*{Notes}

\subsubsection*{File Formats} 
Shapefile (with shp, dbf, shx, and prj)

\subsubsection*{Data Filtering and Processing}
Subset of only the western states in our study

\subsubsection*{Final Variable(s)}

\subsubsection*{Methods}

\begin{enumerate}
\item R\_highways.R script: subset out the western states, arterial (A) and collector (C) roads; sum all A and C road lengths within 100, 250, 500, 1000m buffers of each monitoring location
\end{enumerate}

\subsubsection*{Quality Control}

\subsubsection*{Script Names}

\begin{enumerate}
\item R\_highways.R
\end{enumerate}

\subsubsection*{Original Data File Names}

\begin{enumerate}
\item 
\item 
\end{enumerate}

\subsubsection*{Processed/Cleaned Data File Names}

\begin{enumerate}
\item 
\item 
\end{enumerate}
