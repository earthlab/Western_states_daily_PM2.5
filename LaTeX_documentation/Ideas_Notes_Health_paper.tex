\section{Ideas and Notes for paper}

See this paper for discussion of changes in toxicity of smoke between smoldering/open flame: Kim YH, Tong H, Daniels M, Boykin E, Krantz QT, McGee J, et al.
Cardiopulmonary toxicity of peat wildfire particulate matter and the predictive
utility of precision cut lung slices. Part Fibre Toxicol. 2014;11(1):1–17.

Discuss errors/uncertainties in assigning exposure data in health studies: (references in Linares et al., 2018 \cite{linares_impact_2018}
\begin{enumerate}
\item Weichenthal, S., Kulka, R., Lavigne, E., van Rijswijk, D., Brauer,M., Villeneuve, P.J., Stieb, D.,
Joseph, L., Burnett, R.T., 2017. Biomass burning as a source of ambient fine particulate
air pollution and acute myocardial infarction. Epidemiology 28 (3), 329–337 (May). \url{https://insights.ovid.com/crossref?an=00001648-201705000-00005}

\item Ingebrigtsen, R., Steinsland, I., Cirera, L1., Saez, M., 2015. Spatially misaligned data and the
impact of monitoring network on health effect estimates (Doctoral thesis at NTNU).
In: Ingebrigtsen, R. (Ed.), Bayesian Spatial Modelling of Non-stationary Processes
and Misaligned Data Utilising Markov Properties for Computational Efficiency. Norwegian
University of Science and Technology. 
\end{enumerate}