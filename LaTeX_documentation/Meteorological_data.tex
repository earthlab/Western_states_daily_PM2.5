\subsection{Meteorological Data}
\subsubsection*{Data Source}
\begin{itemize}[nolistsep]
\item \textbf{Contact}
\item \textbf{Citation/Link}
\item \textbf{Data (local)}
\item \textbf{Geographic Extent}
\item \textbf{Temporal Extent}
\item \textbf{Acknowledgment}
\end{itemize}
\subsubsection*{Brief Description}

We will obtain meteorological data from the National Centers for Environmental Prediction (NCEP) North American Regional Reanalysis (NARR) \citep{Mesinger2006,NCEPReanalysis2005} because it includes all of the standard meteorological variables but also has planetary boundary layer height, which has proved to be an important variable for converting AOD to PM\textsubscript{2.5} \citep{liu_estimating_2005}. We will calculate 24-hour averages from 3-hourly data for temperature, relative humidity, sea level pressure, surface pressure, planetary boundary layer height, dew point temperature, precipitation, and the U and V components of wind speed. NARR has 32 km resolution and is available from 1979 onward. 

\subsubsection*{Notes}
\subsubsection*{File Format}
\subsubsection*{Data Filtering and Processing}
\subsubsection*{Final Variable(s)}
\subsubsection*{Methods}
\begin{enumerate}
\item 
\item
\end{enumerate}
\subsubsection*{Quality Control}
\subsubsection*{Script Names}
\begin{enumerate}
\item 
\end{enumerate}
\subsubsection*{Data File Names}
\begin{enumerate}
\item 
\end{enumerate} 