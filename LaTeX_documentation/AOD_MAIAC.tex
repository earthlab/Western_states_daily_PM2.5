\subsection{\texorpdfstring{MAIAC AOD (MCD19A2)}}

\subsubsection*{Data Source}

\begin{itemize}[nolistsep]
\item \textbf{Contact} 
\item \textbf{Citation/Link} https://ladsweb.modaps.eosdis.nasa.gov/archive/allData/6/MCD19A2/?process=ftpAsHttp&path=allData%2f6%2fMCD19A2
\item \textbf{Data (local)} 
\item \textbf{Geographic Extent} 
\item \textbf{Temporal Extent} 
\item \textbf{Acknowledgment} 
\end{itemize}

\subsubsection*{Brief Description}
https://lpdaac.usgs.gov/dataset_discovery/modis/modis_products_table/mcd19a2_v006
"The MCD19A2 Version 6 data product is a MODIS Terra and Aqua combined Multi-angle Implementation of Atmospheric Correction (MAIAC) Land Aerosol Optical Depth (AOD) gridded Level-2 product produced daily at 1 kilometer (km) pixel resolution. The MCD19A2 product provides the atmospheric properties and view geometry used to calculate the MAIAC Land Surface Bidirectional Reflectance Factor (BRF), or surface reflectance, MCD19A1 product."


\subsubsection*{Notes}

\subsubsection*{File Formats} 
HDF; data is in gridded file format

\subsubsection*{Data Filtering and Processing}
Needed to get the lat-lon coordinates separately, from ftp://dataportal.nccs.nasa.gov/DataRelease/MODISTile_lat-lon/ 

Converting to shp and raster files was taking too long (because of the fine resolution), so we used a k-nearest-neighbors approach to estimate the aod values at each monitor location


\subsubsection*{Final Variable(s)}
csv file of monitor locations, dates and aod values

\subsubsection*{Methods}

\begin{enumerate}
\item 
\item
\end{enumerate}

\subsubsection*{Quality Control}

\subsubsection*{Script Names}

\begin{enumerate}
\item 
\end{enumerate}

\subsubsection*{Original Data File Names}

\begin{enumerate}
\item 
\item 
\end{enumerate}

\subsubsection*{Processed/Cleaned Data File Names}

\begin{enumerate}
\item 
\item 
\end{enumerate}
